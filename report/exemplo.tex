% Exemplo de texto que o programa deve gerar
% O nome do arquivo deve ser "membros.tex" e deve estar situado na pasta "dados"
% -------------------------------------------------------------------------------------------------------------------

\begin{membro}
\nome{Daniel Leite}
\tarefas{#STRING1}
\planos{#STRING2}
\comentarios{Coment�rios do jeito que foram escritos, sem modifica��o. Ele pode conter mais de um par�grafo.}
\end{membro}

\begin{membro}
\nome{Rodrigo C�zar}
\tarefas{#STRING1}
\planos{#STRING2}
\comentarios{Ok, eu menti: QUASE sem modifica��o. Aconselharia apenas, se poss�vel, substituir os \n que possivelmente existirem no texto do coment�rio por \par, para que o LaTeX de fato inicie um novo par�grafo.}
\end{membro}

\begin{membro}
\nome{Bianca Martins}
\tarefas{#STRING1}
\planos{#STRING2}
\comentarios{}
\end{membro}
% -------------------------------------------------------------------------------------------------------------------

% FORMATO DA #STRING1
\hline      #tarefa1 & #hora1 & \multirow{#ntarefas}{*}{\centering #horas} \\
\cline{1-2} #tarefa2 & #hora2 & \\
\cline{1-2} #tarefa3 & #hora3 & \\
\cline{1-2} #tarefa4 & #hora4 & \\
\cline{1-2} #tarefa5 & #hora5 & \\
\cline{1-2} #tarefa6 & #hora6 & \\

% FORMATO DA #STRING2
\item #plano1
\item #plano2
\item #plano3
\item #plano3
\item #plano4
\item #plano5

% Substituir #ntarefas pela quantidade total de tarefas da pessoa e #horas pelo total de horas cumpridas

% Todas as quebras de linha nos formatos acima s�o opcionais, a string pode ser em uma linha s� se preferir. Escrevi em v�rias linhas apenas para melhor visualiza��o. Idem para o espa�amento ap�s \hline na primeira linha de STRING1

% Caso a pessoa n�o tenha preenchido algum dos campos e voc� queira explicitar isso, substitua #STRING1 por
\hline \multicolumn{3}{c}{N�o preenchido} \\
% e #STRING 2 por simplesmente
\item N�o preenchido