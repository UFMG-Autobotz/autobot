\begin{membro}
\nome{Bárbara Almeida}
\tarefas{\hline Modificar Simulaçao & 1.0 \\
\hline Estudar plugin da simulalçao & 1.0 \\
\hline Organizaçao da sede & 1.5 \\
\hline Pesquisas sobre robôs de outras equipes e requisitos da plataforma para o trekking e open & 1.0 \\
\hline Reuniao da navegaçao & 1.0 \\
\hline Reuniao da plataforma & 1.0 \\
\hline Estudos e testes de estratégias para a simulaçao do barco & 9.0 \\}{15.5}
\planos{}
\comentarios{}
\end{membro}

\begin{membro}
\nome{Bianca Martins}
\tarefas{}{0}
\planos{}
\comentarios{}
\end{membro}

\begin{membro}
\nome{Bruno Cerqueira}
\tarefas{\hline detalhamento do motor & 3.5 \\
\hline estudo de viabilização Trofeu Solidário & 0.5 \\
\hline registro de material para comunicação & 0.5 \\
\hline confecção de montagem e detalhamento no Inventor & 2.5 \\
\hline confecção de montagem e detalhamento no Inventor & 1.0 \\}{8.0}
\planos{}
\comentarios{}
\end{membro}

\begin{membro}
\nome{Daniel Leite}
\tarefas{\hline Manutenção no barco & 6.5 \\
\hline Organizar a sede & 4.5 \\
\hline Iniciar documentação da {\tt Radio.h} & 1.0 \\}{12.0}
\planos{\item Continuar reparos no barco
\item Testar o código de joystick
\item Continuar documentação da {\tt Radio.h} e testá-la}
\comentarios{Na hora de testar o código do Joystick, vi que alguns aspectos da eletrônica e mecânica do barco não estavam mais funcionando bem. Portanto, tive que repará-los e fazer alguns testes.}
\end{membro}

\begin{membro}
\nome{Elisa Bacelar}
\tarefas{\hline Compilção Vision & 10.0 \\
\hline Reunião da navegação & 1.0 \\
\hline Tirar foto & 1.5 \\}{12.5}
\planos{}
\comentarios{}
\end{membro}

\begin{membro}
\nome{Jonatan Campos}
\tarefas{\hline produção de conteúdo para redes sociais & 2.0 \\
\hline Projeto do VT & 4.0 \\
\hline pesquisar equipes rivais & 2.0 \\
\hline Aprender Python & 1.0 \\}{9.0}
\planos{}
\comentarios{}
\end{membro}

\begin{membro}
\nome{Josué Henrique}
\tarefas{}{0}
\planos{}
\comentarios{}
\end{membro}

\begin{membro}
\nome{Mariana Meireles}
\tarefas{}{0}
\planos{}
\comentarios{}
\end{membro}

\begin{membro}
\nome{Pedro Blanc}
\tarefas{}{0}
\planos{}
\comentarios{}
\end{membro}

\begin{membro}
\nome{Renan Costa}
\tarefas{\hline detalhamento de peças no Inventor & 2.0 \\
\hline transferências de arquivos autoCAD p/ o Inventor & 2.0 \\
\hline início da montagem no inventor & 2.0 \\
\hline redefinição de dimensionamento da placa superior & 1.0 \\
\hline início do desenho do suporte de bateria & 1.0 \\}{8.0}
\planos{}
\comentarios{}
\end{membro}

\begin{membro}
\nome{Rodrigo Cézar}
\tarefas{}{0}
\planos{}
\comentarios{}
\end{membro}

\begin{membro}
\nome{Thiago Lages}
\tarefas{\hline Código de obtenção de variáveis de controle do barco a partir da velocidade de rotação das hélices & 1.0 \\
\hline Código em processing para começar a fazer simulações mais simples de rota do barco & 1.0 \\
\hline Leitura da parte de controle (e não de modelagem) proposto no artigo & 2.0 \\
\hline Reunião navegação & 1.0 \\
\hline Produção de conteúdo para redes sociais & 2.0 \\
\hline Tentativa de pegar a balança para medir forças dos motores no LEx/MACRO & 3.0 \\}{10.0}
\planos{}
\comentarios{}
\end{membro}

\begin{membro}
\nome{Victor Castro}
\tarefas{\hline Concertando o computador da sede - Fracasso! - Necessito de outras peças para teste. Há um problema com a fonte, e instabilidade com CPU ou placa-mãe. Houveram duas inicializações corretas apenas. & 3.5 \\
\hline Tentando reparar a execução do VSS-Simulator-ROS - Sucesso! - Mas ainda existem bugs. & 3.5 \\
\hline A procura de bugs e soluções para o VSS-Simulator-ROS. & 2.5 \\
\hline Reunião da Navegação & 1.0 \\
\hline Produção de conteúdo para redes sociais da Autobotz & 2.0 \\}{12.5}
\planos{\item : Testar uma placa-mãe de CPU fixo encontrada na sede;
\item : Concertar bugs e explorar possíveis melhorias para o VSS-Simulator-ROS
\item : Iniciar a construção da Estratégia para o VSSS}
\comentarios{}
\end{membro}

