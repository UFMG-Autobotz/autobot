\begin{membro}
\nome{Bárbara Almeida}
\tarefas{\hline Pesquisar sobre robôs de outras equipes que competem no Trekking & 3.0 \\
\hline reunião da navegação & 1.0 \\
\hline Inventario de itens da sede + verificar devoluçao de itens & 2.0 \\
\hline Seguir tutoriais do Gazebo, pesquisar sobre atrito no Gazebo, estudar plugins e conexão do Gazebo com ROS, fazer testes com atrito no Gazebo & 20.0 \\}{26.0}
\planos{\item Analizar resultados dos testes de atrito no Gazebo
\item Testar diferentes formas de mover objetos no gazebo (imprimir força, alterar velocidade ou pid / junta ou link)}
\comentarios{Depois de ter bastante dificuldade com testes de formas de simulação no barco no fim de semana decidi voltar um pouco atrás e estudar mais o funcionamento do Gazebo. Aprendi a criar modelos parametrizáveis, a mover objetos por meio de plugins (via velocidade, aceleração ou força),  a pegar informações do modelo (como velocidade e posição),  conectar o plugin com o ROS, salvar os resultados em um arquivo de texto. Usei isso para (apos estudar a forma como o atrito é modelado no Gazebo) fazer testes de atrito que pretendo analisar graficamente na próxima semana.}
\end{membro}

\begin{membro}
\nome{Bianca Martins}
\tarefas{}{0}
\planos{}
\comentarios{}
\end{membro}

\begin{membro}
\nome{Bruno Cerqueira}
\tarefas{}{0}
\planos{}
\comentarios{}
\end{membro}

\begin{membro}
\nome{Daniel Leite}
\tarefas{\hline Manutenção no barco & 2.5 \\}{2.5}
\planos{}
\comentarios{}
\end{membro}

\begin{membro}
\nome{Elisa Bacelar}
\tarefas{\hline Estudo do QtCreator & 2.0 \\
\hline Instalação do ROS Qt Creator Plug-in & 1.0 \\
\hline Utilização do plug-in no Qt Creator com o pacote do Vision & 2.5 \\}{5.5}
\planos{}
\comentarios{}
\end{membro}

\begin{membro}
\nome{Jonatan Campos}
\tarefas{}{0}
\planos{}
\comentarios{}
\end{membro}

\begin{membro}
\nome{Josué Henrique}
\tarefas{}{0}
\planos{}
\comentarios{}
\end{membro}

\begin{membro}
\nome{Mariana Meireles}
\tarefas{}{0}
\planos{}
\comentarios{}
\end{membro}

\begin{membro}
\nome{Pedro Blanc}
\tarefas{}{0}
\planos{}
\comentarios{}
\end{membro}

\begin{membro}
\nome{Renan Costa}
\tarefas{}{0}
\planos{}
\comentarios{}
\end{membro}

\begin{membro}
\nome{Rodrigo Cézar}
\tarefas{}{0}
\planos{}
\comentarios{}
\end{membro}

\begin{membro}
\nome{Thiago Lages}
\tarefas{}{0}
\planos{}
\comentarios{}
\end{membro}

\begin{membro}
\nome{Victor Castro}
\tarefas{\hline Otimizado o VSS-SDK-ROS em geral - Redução de 100% de uso de CPU para 20% ao executar o Viewer, Simulator e Sample simultaneamente. & 2.0 \\
\hline Debugando o VSS-Simulator-ROS & 8.0 \\
\hline Reunião da Navegação & 1.0 \\}{11.0}
\planos{\item Explorar soluções para o VSS-Simulator-ROS}
\comentarios{Nesta semana, rodei a estratégia em C++ desenvolvida no Trainee 2017 no VSS-Simulator-ROS a fim de testá-lo. Pensei que com a solução do grande bug que causava o desaparecimento dos robôs, poderiamos começar a desenvolver uma estratégia sólida. Porém, foram encontrados bugs com as mesmas proporções do anterior. Então, novamente gastarei mais algumas horas tentando resolver na próxima semana, e em caso de fracasso, iniciarei o desenvolvimento de um novo simulador.}
\end{membro}

