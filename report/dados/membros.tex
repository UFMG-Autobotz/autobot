\begin{membro}
\nome{Bárbara Almeida}
\tarefas{\hline Seguir tutoriais do Gazebo & 1.5 \\}{1.5}
\planos{}
\comentarios{}
\end{membro}

\begin{membro}
\nome{Bianca Martins}
\tarefas{}{0}
\planos{}
\comentarios{}
\end{membro}

\begin{membro}
\nome{Bruno Cerqueira}
\tarefas{}{0}
\planos{}
\comentarios{}
\end{membro}

\begin{membro}
\nome{Daniel Leite}
\tarefas{\hline Reparar o Circuito LiPo & 2.0 \\
\hline Testar o Circuito LiPo & 2.0 \\}{4.0}
\planos{}
\comentarios{}
\end{membro}

\begin{membro}
\nome{Elisa Bacelar}
\tarefas{}{0}
\planos{}
\comentarios{}
\end{membro}

\begin{membro}
\nome{Jonatan Campos}
\tarefas{}{0}
\planos{}
\comentarios{}
\end{membro}

\begin{membro}
\nome{Josué Henrique}
\tarefas{\hline Tentativa de converter arquivos do Inventor (.iam) e do SolidWorks (.sldptr) para URDF & 3.0 \\}{3.0}
\planos{}
\comentarios{}
\end{membro}

\begin{membro}
\nome{Mariana Meireles}
\tarefas{}{0}
\planos{}
\comentarios{}
\end{membro}

\begin{membro}
\nome{Pedro Blanc}
\tarefas{}{0}
\planos{}
\comentarios{}
\end{membro}

\begin{membro}
\nome{Renan Costa}
\tarefas{}{0}
\planos{}
\comentarios{}
\end{membro}

\begin{membro}
\nome{Rodrigo Cézar}
\tarefas{\hline Consertos do Bot & 0.5 \\}{0.5}
\planos{}
\comentarios{}
\end{membro}

\begin{membro}
\nome{Thiago Lages}
\tarefas{\hline leitura e entendimento do artigo de controle de hovercraft providenciado pelo MACRO & 2.0 \\
\hline escrita de um documento resumo dos fundamentos e princípios matemáticos utilizados no artigo, verificacão de alguns cálculos e relacao com o que teremos no nosso barco (quais sensores fornecerão quais dados, quais são as variáveis de controle, como as controlaremos, etc) & 1.5 \\
\hline pesquisa sobre efeito Coriolis (textos e videos) e artigo 'Guidance and Control of Ocean Vehicles', T. I.  Fossen, que fala sobre modelagem de veiculos oceânicos; fazendo algumas simplificacões, desconsiderando efeitos do vento e ondas, e movimentos como roll, pitch (rotacões nos eixos X e Y) e heave (movimento up/down no eixo Z). Sua importância se dá no fato de que é citado no artigo fornecido pelo MACRO para a equipe (cuja modelagem é praticamente toda baseada no artigo do Fossen). Além disso, pensei e discuti com os membros Josué, Daniel e Jonatan sobre medir forca de cada um dos propulsores. Pensamos em medir de alguma maneira (arduino, encoders) a velocidade das hélices e comparar com as tabelas de velocidades do datasheet, utilizar um equipamento do Lex próprio para esse tipo de medicao, entre outros. Fiz cálculos relacionando a forca dos propulsores com as variáveis que temos que levar em conta na modelagem, como as velocidades 'surge', 'sway' do barco e o torque em torno do eixo Z. Cálculo aproxmado do momento de inércia do barco, para relacionar torque e aceleracão angular. & 4.0 \\}{7.5}
\planos{}
\comentarios{}
\end{membro}

\begin{membro}
\nome{Victor Castro}
\tarefas{}{0}
\planos{}
\comentarios{}
\end{membro}

