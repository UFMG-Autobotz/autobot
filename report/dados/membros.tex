\begin{membro}
\nome{Bárbara Almeida}
\tarefas{\hline Compilar Simulator & 1.0 \\
\hline Rodar Viewer & 1.0 \\
\hline Modificar e compilar Sample & 1.5 \\
\hline Mesclar modulos & 1.0 \\
\hline Organizar sede & 2.0 \\
\hline Compilar Vision & 2.0 \\
\hline Tentar rodar simulator & 4.5 \\
\hline Reunioes semana 31 & 5.0 \\
\hline Documentação VSSS & 2.0 \\
\hline Reuniões semana 32 & 5.0 \\}{25.0}
\planos{\item Aprender a usar o Gazebo}
\comentarios{: Como o bot parou de funcionar no fim de semana, contém atividades das semanas 31 e 32}
\end{membro}

\begin{membro}
\nome{Bianca Martins}
\tarefas{\hline Pesquisas inicias sobre utilização do Galileo & 4.0 \\
\hline Reuniões semana 31 & 5.0 \\
\hline Reuniões semana 32 & 2.5 \\
\hline Montar circuito de pwm, receptor e transmissor infravermelho no multisim, na protoboard e testar & 9.0 \\}{20.5}
\planos{\item Continuar pesquisa sobre Galileo e realizar testes; Documentar resultados}
\comentarios{}
\end{membro}

\begin{membro}
\nome{Bruno Cerqueira}
\tarefas{\hline Elaboração do Plano de Ação da Comunicação & 3.0 \\
\hline Mapeamento de Emails Professores & 1.0 \\
\hline dimensionamento e desenho de peças no Inventor & 4.0 \\
\hline desenho de peça no MDF & 1.0 \\
\hline reunião semanal & 2.0 \\
\hline Estudo da proposta do chassi de 3 rodas e de conceitos básicos & 1.0 \\}{12.0}
\planos{}
\comentarios{}
\end{membro}

\begin{membro}
\nome{Daniel Leite}
\tarefas{\hline Diagnosticar erros na placa do Circuito LiPo & 3.0 \\
\hline Testes no código e programação do PIC do circuito LiPo & 5.0 \\
\hline Reuniões semana 31 & 5.0 \\
\hline Reunião semana 32 & 2.5 \\
\hline Correções e melhorias na biblioteca de Rádio & 4.5 \\
\hline Tentar consertar a fonte simétrica & 1.0 \\}{21.0}
\planos{\item Terminar testes no Circuito LiPo
\item Começar a trabalhar na operação do barco via controle remoto
\item Testar biblioteca de rádio e continuar melhorias}
\comentarios{Os testes no Circuito LiPo foram comprometidos pelo fato de a fonte simétrica ter queimado. Tentarei consertá-la ou procurar alguma maneira de como testá-lo sem precisar dela.\par Surgiu a necessidade de se utilizarem os rádios da sede para o futebol, de forma que esta é uma boa hora para testar a biblioteca {\tt Radio.h} feita em março deste ano. Comecei a relembrar os conceitos do módulo de rádio e a corrigir alguns erros no código. Darei continuidade à correção dos erros e aos testes, mas esta não é a minha tarefa primária. Trabalharei com isso em paralelo ao Circuito LiPo e, após terminá-lo, em paralelo à operação do barco via controle remoto.}
\end{membro}

\begin{membro}
\nome{Elisa Bacelar}
\tarefas{\hline Compilação do vision & 8.0 \\
\hline Criação do qt package do vision & 8.0 \\
\hline Ajudar victor com funcionamento do viewer & 3.0 \\
\hline Arrumação da sede & 3.0 \\
\hline Instalação das ferramentas no novo computador & 2.0 \\}{24.0}
\planos{}
\comentarios{}
\end{membro}

\begin{membro}
\nome{Jonatan Campos}
\tarefas{}{0}
\planos{}
\comentarios{}
\end{membro}

\begin{membro}
\nome{Josué Henrique}
\tarefas{}{0}
\planos{}
\comentarios{}
\end{membro}

\begin{membro}
\nome{Mariana Meireles}
\tarefas{\hline Aprendendo a usar o Instrumentino & 2.5 \\
\hline Leitura da bateria & 1.5 \\}{4.0}
\planos{}
\comentarios{}
\end{membro}

\begin{membro}
\nome{Pedro Blanc}
\tarefas{\hline Planilha de Compras & 4.0 \\
\hline Reunião de Coordenadores & 1.0 \\}{5.0}
\planos{}
\comentarios{}
\end{membro}

\begin{membro}
\nome{Renan Costa}
\tarefas{}{0}
\planos{}
\comentarios{}
\end{membro}

\begin{membro}
\nome{Rodrigo Cézar}
\tarefas{\hline Reunião semanal da equipe (11/08) & 1.0 \\}{1.0}
\planos{}
\comentarios{}
\end{membro}

\begin{membro}
\nome{Thiago Lages}
\tarefas{\hline Pesquisar a respeito de filtro de Kalman e filtro de Wiener, assistir video aulas, videos de implementações dos filtros, exemplos de aplicações. Enviar e-mail para um aluno de mestrado UFSJ/CEFET a respeito de sua implementação em MATLAB, procurando saber sobre como foi feita aquisição de dados e plot de gráfico em tempo real no MATLAB, além de dicas e o que ele achava a respeito da nossa implementação para resolver o problema de localização do barco. Algumas tentativas de implementação da aquisição em tempo real de dados da IMU em MATLAB, sem sucesso. A aquisição foi feita na própria Arduino IDE. & 12.0 \\
\hline Procura de implementações de código para teste da IMU. Leitura, entendimento e debug do código existente da equipe para fazer funcionar o teste da mesma. Leitura, entendimento e debug do código em Processing para visualização dos resultados de localização espacial da IMU. Variados testes como obtenção inicial de dados, X amostragens dos sinais iniciais pra melhor calibragem dos sensores.  Obtenção de algumas equações para os as angulações dos motores do barco, que relacionam os dados de entrada que temos, como posição (pelos ultrassonicos) e valores dos outros sensores, e a saida desejada (angulacao dos motores) & 12.0 \\}{24.0}
\planos{}
\comentarios{}
\end{membro}

\begin{membro}
\nome{Victor Castro}
\tarefas{}{0}
\planos{}
\comentarios{}
\end{membro}

