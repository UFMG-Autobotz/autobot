\begin{membro}
\nome{Bárbara Almeida}
\tarefas{}{0}
\planos{}
\comentarios{}
\end{membro}

\begin{membro}
\nome{Bianca Martins}
\tarefas{\hline Resolver problemas da alimentação do Galileo e montar regulador de tensão & 3.5 \\
\hline Pesquisar e iniciar a instalação do debian no galileo & 4.5 \\
\hline Reuniões geral e Eletrônica & 2.5 \\
\hline Brainstorm sobre melhorias no projeto da HAMS e início do layout & 1.5 \\}{12.0}
\planos{\item Resolver os problemas da instalação do debian
\item Terminar o layout revisado da HAMS}
\comentarios{}
\end{membro}

\begin{membro}
\nome{Bruno Cerqueira}
\tarefas{\hline desenho de peças & 3.0 \\
\hline começo da estrutura do protótipo em MDF & 1.0 \\
\hline reunião semanal & 2.0 \\
\hline detalhamento de peças & 1.0 \\
\hline detalhamento de peças & 1.0 \\
\hline reunião da Mecânica & 1.0 \\
\hline detalhamento do motor & 1.0 \\
\hline detalhamento do motor & 1.0 \\
\hline elaboração do cronograma da Mecânica VSS & 1.0 \\}{12.0}
\planos{}
\comentarios{}
\end{membro}

\begin{membro}
\nome{Daniel Leite}
\tarefas{\hline Testar Circuito LiPo & 8.0 \\
\hline Diagnosticar e reparar Circuito LiPo & 5.0 \\
\hline Pesquisar funcionamento de encoder magnético para o VT & 1.0 \\
\hline Organizar LEDs da sede & 0.5 \\
\hline Reunião semanal & 2.5 \\}{17.0}
\planos{\item Documentar Circuito LiPo
\item Projetar melhorias para o Circuito LiPo
\item Testar código de joystick utilizando o barco
\item Estudar a implementação de encoder magnético no Veículo Terrestre}
\comentarios{Agora que a versão atual do Circuito LiPo funciona, falta implementar as melhorias para a nova versão, que será usada no Veículo Terrestre.\par Não houve tempo de trabalhar na operação do barco via joystick, mas dedicarei esta próxima semana a isso.}
\end{membro}

\begin{membro}
\nome{Elisa Bacelar}
\tarefas{\hline Compilação do package do Vision & 8.5 \\
\hline Reuniões semanais & 3.5 \\}{12.0}
\planos{}
\comentarios{}
\end{membro}

\begin{membro}
\nome{Jonatan Campos}
\tarefas{\hline Projetar peças do VT & 5.0 \\
\hline Reunião Autobotz & 2.0 \\
\hline Reunião da Mecânica & 0.5 \\
\hline Fazer conectores & 0.5 \\
\hline Brainstorm sobre o VT & 2.5 \\}{10.5}
\planos{}
\comentarios{}
\end{membro}

\begin{membro}
\nome{Josué Henrique}
\tarefas{\hline Calculo referentes ao braço da plataforma terrestre & 1.0 \\
\hline Busca por peças necessárias para a mecânica de acordo com as novas específicaçoes & 2.0 \\
\hline Brainstorming sobre alterações necessárias na plataforma terrestre. & 2.0 \\
\hline Reunião semanal da mecânica & 0.5 \\
\hline Apresentação semanal de projetos & 2.0 \\
\hline Desmontagem da propulsão e separação de material necessário para testes no lex & 1.5 \\}{9.0}
\planos{}
\comentarios{}
\end{membro}

\begin{membro}
\nome{Mariana Meireles}
\tarefas{\hline estudando arduino & 1.0 \\
\hline estudando cpp & 2.0 \\
\hline fazendo circuito e entendendo o L298 & 0.5 \\
\hline reunião da equipe & 2.5 \\
\hline testando biblioteca criada por mim na ponte h & 2.0 \\
\hline concertando problemas derivados da biblioteca & 4.0 \\}{12.0}
\planos{}
\comentarios{}
\end{membro}

\begin{membro}
\nome{Pedro Blanc}
\tarefas{\hline Preparar apresentações de aulas de calouros. Apresentar. Ensaiar com Bárbara, Mariana e Bianca. & 4.0 \\
\hline Pesquisar sobre opções de reviver a wiki & 1.0 \\
\hline Reunião da eletrônica e apresentações internas do barco e da esteira & 3.0 \\
\hline Retirar componentes de placas da sucata para jogar as placas no lixo & 4.0 \\
\hline Testar código do barco de leitura dos ultrassons & 1.0 \\}{13.0}
\planos{}
\comentarios{}
\end{membro}

\begin{membro}
\nome{Renan Costa}
\tarefas{\hline reunião mecânica & 1.0 \\
\hline reunião geral & 2.0 \\
\hline montagem inicial do protótipo & 1.0 \\
\hline elaboração do cronograma do VSS & 1.0 \\
\hline finalização do suporte da bateria e alterações na placa superior & 2.0 \\
\hline exportar arquivos do autocad para o inventor & 1.0 \\
\hline desenho das rodas e da bateria no inventor & 2.5 \\
\hline início de projeto da peça de chute & 1.5 \\}{12.0}
\planos{}
\comentarios{}
\end{membro}

\begin{membro}
\nome{Rodrigo Cézar}
\tarefas{\hline Reunião da Navegação & 0.5 \\
\hline Reunião da Equipe & 2.5 \\
\hline Conversas individuais com membros (RH) & 0.5 \\
\hline Tutoriais de gazebo & 1.0 \\
\hline Ajustes na modelagem das peças do VT & 3.0 \\
\hline Fazer uma versão preliminar do plugin para interagir com o modelo via ROS (escrever o código e consertar erros de execução) & 9.0 \\}{16.5}
\planos{}
\comentarios{}
\end{membro}

\begin{membro}
\nome{Thiago Lages}
\tarefas{\hline Reunião Navegação & 1.0 \\
\hline Reunião Equipe & 2.5 \\
\hline Desmontar propulsores para medição no LEX & 0.5 \\
\hline Preparação dos slides para apresentação semanal & 1.5 \\
\hline Reler partes do artigo, entender variáveis matriciais que estavam faltando, terminar de calcular/pensar maneiras de calcular momento de inércia do barco & 3.5 \\
\hline Começar a escrever código modularizado que receberá dados de velocidade do motor do barco, retornará valores de força gerado por ele (depois das medições), e fornecerá esse dado para a função da modelagem do barco, que determinará sua posição (x,y) e orientação (phi) & 2.0 \\
\hline Olhar possibilidade de trabalhar com as modelagens feitas utilizando Gazebo & 0.5 \\
\hline Olhar polls possíveis para se aplicar na equipe para escolha de pautas & 0.5 \\}{12.0}
\planos{}
\comentarios{}
\end{membro}

\begin{membro}
\nome{Victor Castro}
\tarefas{\hline Investigando o erro de execução do VSS-Simulator-ROS & 8.25 \\
\hline Reunião da Navegação & 0.5 \\
\hline Reunião Geral & 2.5 \\
\hline Pesquisa inicial a respeito da construção de um novo simulador para o VSS-SDK-ROS & 0.5 \\}{11.75}
\planos{\item : Última semana de trabalho com o simulador atual; em caso de fracasso, iniciarei a construção de um novo.}
\comentarios{}
\end{membro}

