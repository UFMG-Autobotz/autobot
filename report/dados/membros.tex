\begin{membro}
\nome{Bárbara Almeida}
\tarefas{\hline Estudar controle & 8.0 \\
\hline Reuniao sobre controle do barco & 3.0 \\
\hline Calculos para controle de versao simplificada do barco & 1.0 \\
\hline Desenvolvimento de plugin para comunicaçao da câmera do Gazebo com o ROS & 3.0 \\
\hline Inicio da documentaçao do plugin para controle de juntas (finalizado na semana anterior) & 2.0 \\
\hline Inicio do desenvolvimento de plugin para receber informaçoes sobre links no Gazebo & 2.0 \\
\hline Testes para simulação do futebol no Gazebo (projeto de robô simples do futebol, descobrir como adicionar textura à parte superior do robô, teste com os plugins genéricos desenvolvidos anteriormente para controle das rodas e receber imagem da câmera) & 2.0 \\}{21.0}
\planos{\item : Terminar plugins do Gazebo
\item : Terminar documentaçao dos plugins
\item :Estudar controle
\item :Criar modelo simplificado do barco no Gazebo}
\comentarios{: Para esta semana estava planejado iniciar o controle do barco, para isso estudei um pouco sobre controle durante a semana e na sexta-feira eu, o Rodrigo e o Thiago nos reunimos para discutir o que estudamos e como faremos o controle. \par Como nao consegui terminar os plugins que estavam planejados para a semana passada, também continuei trabalhando com o Gazebo essa semana, ainda esta faltando terminar o plugin para receber infomaçoes sobre links e terminar as documentações.\par Como houve problemas para utilizar a simulação antiga do futebol (feita no V-Rep), também testei a possibilidade de se utilizar o Gazebo para simular o futebol. Ja é possível projetar a simulação do futebol com o que temos até agora, mas eu quero tentar mudar a textura da tag do robô via plugin (para conseguir adicionar 6 instâncias do mesmo modelo e modificar apenas a tag, ao invés de criar 6 modelos com apenas a tag de diferença)}
\end{membro}

\begin{membro}
\nome{Bianca Martins}
\tarefas{}{0}
\planos{}
\comentarios{}
\end{membro}

\begin{membro}
\nome{Bruno Cerqueira}
\tarefas{}{0}
\planos{}
\comentarios{}
\end{membro}

\begin{membro}
\nome{Daniel Leite}
\tarefas{}{0}
\planos{}
\comentarios{}
\end{membro}

\begin{membro}
\nome{Elisa Bacelar}
\tarefas{}{0}
\planos{}
\comentarios{}
\end{membro}

\begin{membro}
\nome{Jonatan Campos}
\tarefas{}{0}
\planos{}
\comentarios{}
\end{membro}

\begin{membro}
\nome{Josué Henrique}
\tarefas{}{0}
\planos{}
\comentarios{}
\end{membro}

\begin{membro}
\nome{Mariana Meireles}
\tarefas{}{0}
\planos{}
\comentarios{}
\end{membro}

\begin{membro}
\nome{Pedro Blanc}
\tarefas{}{0}
\planos{}
\comentarios{}
\end{membro}

\begin{membro}
\nome{Renan Costa}
\tarefas{\hline alterações e melhoria no desenho do chutador & 3.0 \\
\hline desenho de placas eletrônicas & 2.0 \\
\hline correções em medidas & 2.0 \\
\hline Aprendendo Python & 1.0 \\}{8.0}
\planos{}
\comentarios{Fiquei doente na sexta, o que prejudicou o cumprimento das 12 horas semanais}
\end{membro}

\begin{membro}
\nome{Rodrigo Cézar}
\tarefas{}{0}
\planos{}
\comentarios{}
\end{membro}

\begin{membro}
\nome{Thiago Lages}
\tarefas{\hline Leitura e vídeos sobre espaços de estados & 3.0 \\
\hline Leitura e vídeos sobre modelagens de sistemas físicos & 2.0 \\
\hline Revisão de conceitos de geometria analítica & 1.0 \\
\hline Leitura sobre modelos lineares e não lineares, Fossen. & 2.0 \\
\hline Leitura e vídeos sobre conceitos de controle & 2.0 \\
\hline Discussão com membros da equipe acerca de situações simples que poderiamos tentar modelar, passar para espaço de estados, criar um controlador simples e simular. Definição de topologia de controlador, planta, sensores, entradas e saídas de cada um. & 4.0 \\}{14.0}
\planos{}
\comentarios{}
\end{membro}

\begin{membro}
\nome{Victor Castro}
\tarefas{\hline Teste do V-REP + ROS Indigo & 2.0 \\
\hline Tentativa de integrar V-REP + ROS Kinetic & 4.0 \\
\hline Decisão de continuar com o projeto antigo do VSSS & 2.0 \\
\hline Manutenção dos computadores (meu e da Elisa) & 2.5 \\
\hline Pesquisa de simuladores compatíveis com o ROS para ser usado no VSSS & 2.5 \\}{13.0}
\planos{}
\comentarios{Foi decidido abandonar o projeto antigo devido a falta de documentação e erros de compatibilidade com novas versões do ROS e do V-REP}
\end{membro}

